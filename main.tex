%%%%%%%%%%%%%%%%%%%%%%%%%%%%%%%%%%%%%%%%%%%%%%%%%%%%%%%%%
%
%    AGH University of Krakow Beamer Theme Presentation
%    Prezentacja systemu: System do śledzenia przebiegu eksperymentów labolatoryjnych
%    Autorzy: Oliwia Rewer, Paulina Wór, Mateusz Woźniak
%
% opis aplikacji - zastosowanie, użytkownicy, role,
% opis walidacji wprowadzanych danych,
% dane - skąd były wzięte i w jaki sposób zostały wprowadzane,
% omówienie wyglądu aplikacji - style CSS,
% dynamiczna zawartość - skrypty JS, generowanie mediów przez serwer,
% najciekawsze elementy aplikacji,
% sposób testowania aplikacji.
%
%
%%%%%%%%%%%%%%%%%%%%%%%%%%%%%%%%%%%%%%%%%%%%%%%%%%%%%%%%%

\documentclass[polish,aspectratio=1610]{beamer}
\usetheme[parttitle=date]{AGH} % Bez opcji 'dark' dla białego tła
\usepackage[utf8]{inputenc}
\usepackage[T1]{fontenc}
\usepackage{graphicx}
\usepackage{hyperref}
\usepackage[polish]{babel}

\title{Prezentacja systemu:\\System do śledzenia przebiegu eksperymentów labolatoryjnych}
\author{Rewer, Wór, Woźniak}

\date{}

\begin{document}

\maketitle

%%%%%%%%%%%%%%%%%%%%%%%%%%%%%%%%%%%%%%%%%%%%%%
\section{Wprowadzenie}
%%%%%%%%%%%%%%%%%%%%%%%%%%%%%%%%%%%%%%%%%%%%%%
\begin{frame}{Wprowadzenie}
    \begin{itemize}
        \item Celem projektu jest stworzenie automatycznego systemu monitorowania sali szpitalnej.
        \item System wykorzystuje algorytm YOLO uruchomiony na Raspberry Pi 4B z podłączoną kamerą.
        \item Oprogramowanie udostępnia wyniki przez REST API oraz prezentuje je na panelu webowym.
        \item Przewidywana dokładność detekcji: co najmniej 70\% przy latencji poniżej 1 sekundy.
    \end{itemize}
\end{frame}

\end{document}
